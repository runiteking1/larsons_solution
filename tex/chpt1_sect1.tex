\setcounter{Exercise}{5}
\begin{Exercise}
We just have to show that there exists an infinite number of the digit 6 in the sequence, and not to find a closed-form expression of it. If we extend the sequence, it should be of the following form $2,7,1,4,7,4,2,8,2,8,8,1,6,1,6,1,6,6,4,8,6,6,6,6,\ldots$ 

The 4 consecutive 6's seems intriguing, so lets study what happens when we expand it out according to our rule: $3,6,3,6,3,6,3,6$. 
This is good! Four consecutive 6's produces four more 6's, but the next iteration doesn't seem so promising: $1,8,1,8,1,8,1,8,1,8,1,8,1,8$. There are no 6's this time. 

Let's not give up so early, here are the next few iterations (with truncation): 
\begin{align}
8,8,\ldots, 8 \rightarrow 6,4,6,4,\ldots,6,4 \rightarrow 2,4,2,4,\ldots, 2,4 \rightarrow 8,8,\ldots, 8. 
\end{align}
This is great news! We actually found a cycle which gaurentees us more 6's if we follow it, and hence the exercise is proven (informally). \qed
\end{Exercise}

\begin{Exercise}
All prime numbers satisfy the property that the first $n-1$ digit of $S_n$ are $n$. This pattern is fairly obvious if one
starts to extend the $S_n$ sequences out a bit more. The first conjecture might be that all odd numbers satisfy the property,
but the case of $S_9$ is a counter-example. 

The proof is as follows: we first note that if $n$ is in $S_n$, then it will be incremented to $n+1$ in $S_{n+1}$. 
Next, note that if an element of the sequence is a prime number, it can only increment by one if $n$ (of the index $S_n$)
 is the same prime number.

If we put these two facts together, we see that given a prime $p$, the first $p-1$ numbers will increment along with $S_n$ by
the first statement to reach $p$ as $n \rightarrow p$. 
The speed of incrementation might be different, but we really don't care about that by the second statement. 
If an element of the sequence reached $p$ already, then it'll stay at $p$ since no other numbers will divide it.

Note that the $p, p+1, p+2, \ldots$th numbers will be greater than $p$ by the $S_2$. \qed
\end{Exercise}

\begin{Exercise}
The intuition for this problem is quite easy to grasp as the patterns are laid out before our eyes.
I believe the hard part is putting that solution to words. 
There's one thing that I used in proving (albeit less rigorous than I like) this problem, which is that the 
alternate sum of binomial coefficients is 0. This is a commonly used fact, and is quite useful at times. 

Start with $n=3$ for a concrete example. The following sequence is an acceptable solution: 
\begin{align}
	01212123
	\label{eqn:118}
\end{align}
where each digit signifies the index in Pascal's triangle. We start with the null set, then pick any set with one element 
and alternate with subsets of two elements and so on.
The algorithm is a greedy one, and we wish to show it satisfies all the conditions.

Condition one is easy to show, but why are we guaranteed that conditions (ii) and (iii) of the problem is satisfied?
For condition (iii), note that if we simply alternate in the order of the elements, we will always be able to alternate between
a subset of size $m$ and $m+1$. So in our small example, we would add the subset of just element 1, then add element 2, take 
away element 1, add element 3 and so on.

To see that we do indeed use every subset, we need to put our algorithm into numbers. Using our small example again,
we start with the standard Pascal triangle row: 
\begin{align}
	1\| 3 \| 3 \| 1.
\end{align}
This means our sequence needs to have one 0, three 1's, three 2's and one 3 (like in equation~\ref{eqn:118}).
If we alternate between the $\binom{3}{0}$ and $\binom{3}{1}$ we will end up with this:
\begin{align}
	0\| \dot{2} \| 3 \| 1
\end{align}
where the dot indicates the size of the subset of the end of our sequence. 
Alternating between $\binom{3}{1}$ and $\binom{3}{2}$ will result in 
\begin{align}
	0\| 0 \| \dot{0} \| 1.
\end{align}
This configuration means that our sequence's last element is of size 2, and we have one last subset of size 3 to take care of.
At the end, all our numbers will be 0; this is what we want to capture in our proof.

If one continues this pattern in general for all $n$, we will have the following equation
\begin{align}
	{n \choose n} - \left( {n\choose n-1} - \left( {n \choose n-2} - \cdots \left( {n \choose 1} - {n \choose 0} \right) +1 \right) + \cdots +1 \right)
\end{align}
I realize that's a big jump, but try to convince yourself that if you work out the pattern, it's what you see. 
There's really no work left, as that equation simplifies to the alternating sum of binomial coefficients which is 0! 
This proves\footnote{Not that rigorous I suppose} that our algorithm will satisfy all three conditions. \qed
\end{Exercise}

\begin{Exercise}
Two immediate ways to solve this: 
\begin{itemize}
	\item Look at the Pascal triangle in terms of even-odd. You'll quickly notice a pattern that the odd binomial
	coefficients resemble a Sierpinski triangle. It'll be extremely tedious, but I think given enough patience and
	organization one can count them in terms of the pattern.
	\item Note the following pattern starting from row 1: $1,2,2,4,2,4,4,8,2$. Instead of looking at the powers of 2,
	maybe the exponents will give us more insight: $0,1,1,2,1,2,2,3,1$. An \emph{extremely} astute observer will
	note that it's the number of ones in the binary expansion of the row number! For example, row 3 (1,3,3,1) is 11 in 
	binary so the number of odd coefficients is $2^2$ for the number of 1s in the binary representation. 

	I remembered this trick back in high school, so don't fret if you can't figure this out. The theorem which this is based
	on is Lucas' theorem.\footnote{\url{http://en.wikipedia.org/wiki/Lucas'\_theorem}} \qed
\end{itemize}
\end{Exercise}

\begin{Exercise}
	I'm not sure how to cleverly solve this one. My approach was to look at the equations in various mods (i.e. 2, 4, and 8) 
	and see what I can clobber together. I couldn't prove that all such primes are able to be expressed in the forms though.
\end{Exercise}

\begin{Exercise}
	The conjecture should be fairly simple to make; try writing a few terms of the sequence starting with $1/2, 1, 2, -1$. 
	The limit seems to be 1, which is easily proved with a technique that's covered in 7.6.4 as mentioned in the problem. 

	In order to prove this though, one can approach the problem like 1.1.3. We note that $a_2 = \frac{1}{2-a_1}$. 
	Continuing the series:
	\begin{align}
		a_3 &= \frac{1}{2-a_2} = \frac{1}{2 - \frac{1}{2-a_1}} = \frac{2-a_1}{3-2a_1} \\
		a_4 &= \frac{3-2a_1}{4-3a_1} \\
		a_5 &= \frac{4-3a_1}{5-4a_1}
	\end{align}
	The patterns is quite obvious now: 
	\begin{align}
		a_n &= \frac{(n-1) - (n-2)a_1}{n - (n-1)a_1}
	\end{align}
	This can be proven by induction quite easily (but a pain to type out). The limit of that as $n$ approaches infinity
	is 1. \qed
\end{Exercise}

\begin{Exercise}
	This took a considerable amount of time to proof, from just getting a feel of the problem to finally seeing a solution.
	The key was the expression $(a\cdot b) \cdot ((a\cdot b) \cdot b)$. Note that we can express this in two forms! 
	It can equal $b$ (by considering $a\cdot b$ together) or $(a\cdot b) \cdot a$ (by considering the right most parenthesis first).

	Now that we have $b = (a\cdot b) \cdot a$, we simply apply $a$ to both sides to get $b \cdot a = ((a\cdot b) \cdot a) \cdot a = a \cdot b$. \qed
\end{Exercise}
