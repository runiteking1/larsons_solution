\setcounter{Exercise}{7}
\begin{Exercise}
	The hint is a pretty big one; if we use it, we arrive at a polynomial of the form $(y+1)^7 - 2(y+1)^5 + 10(y+1)^2 - 1$.
	If we naively expand everything, we will see that the corresponding equation is of the form
	\begin{align}
		y^7 + 7y^6 + 19y^5 + 25y^4 + 15y^3 + 11y^2 + 17y + 18
	\end{align}
	This polynomial has coefficients bigger than one, hence for any positive $y$, our polynomial will be greater than 0.\qed
\end{Exercise}

\begin{Exercise}
	The simplest way to do this is to recast this problem as a combinatorics problem. 
	Consider viewing the number $n$ as $n$ individual stones and consider the gaps in between.
	There is a one-to-one correspondence between choosing whether we want to select the gap (thus partitioning up the stones)
	and the quantity we are looking for. 

	For example, in the case of $n=3$, the case of 1+2 would be to select the first gap and not select the second. 
	The result is immediate after this recasting. \qed
\end{Exercise}

\begin{Exercise}
	Once again, view this an a combinatorics problem, but more sticks-and-balls type.\footnote{\url{http://www.math.illinois.edu/~hildebr/putnam/training/combinatorial13-2sol.pdf}} Imagine expressing the number 10 as ten stones and having four sticks
	which can partition the stones into five. Since we care about order, then it's simply the number of ways we can order the
	the balls and four sticks $\binom{10+4}{4}$. 

	See the footnotes for a bit more detailed solution on how the whole sticks-and-ball problems work.\qed
\end{Exercise}

\begin{Exercise}
	Using the hint, we can view this problem as a geometry exercise.
	The first expression $x^2 = y^2$ is simply two intersecting lines
	at the origin, while the second expression is that of a circle
	centered at $(a, 0)$ with radius 1. 

	From this, it is easy to see that there are four solutions when
	$\{-\sqrt{2} < a < \sqrt{2}\} - {-1, 1}$, three solutions at ${1,-1}$, two solutions $a=\pm\sqrt{2}$, and zero solutions elsewhere. Nowhere is there only one solution
	due to symmetry of the problem. \qed
\end{Exercise}

\begin{Exercise}
	TODO
\end{Exercise}

\begin{Exercise}
	TODO
\end{Exercise}

\begin{Exercise}
	First, some notation.
	Let $a,b,c$ be the sides which correspond to $p_1, p_2, p_3$ respectively. 
	We use the hint:
	\begin{align}
		2A &= ap_1 = bp_2 = cp_3 \\
		2A &= r(a + b + c)
	\end{align}

	The first equation comes from basic geometry, while the second comes form noticing that the incircle is tangent to each side; hence a right angle with each side.

	Now, we simply substitue the $a = \frac{2A}{p_1}$ into the second equation, along with $b, c$.
	Cancelling the $2A$, from both sides of the equation and moving $r$ to the other side gives us the answer. \qed
\end{Exercise}
