\setcounter{Exercise}{5}

\begin{Exercise}
	\begin{enumerate}
		\item By symmetry, we can simply expand the first term and know what the other terms will be like.
		Note that by expanding the $x^2y$ element in the first parenthesis, we have $x^3y^3, x^2y^2z^2$ and $x^4yz$.
		Thus, we will have $3x^3y^2z^2 + x^3y^3 + x^3z^3 + y^3z^3 + x^4yz + xy^4z + xyz^4$.

		\item There's a typo in my version of the book.
		It should be $x^5 + y^5 + z^5$ instead of $x^2 + y^5 + z^5$.

		If we actually use the proposed substitution, this turns into a tedious problem in algebra.
		Fortunately, symmetry gives us a few shortcuts.
		First, notice that the leftmost group gives $x^2 + xy + y^2$ and the middle group is the binomial expansion of $(-x-y)^5$ without the initial $x^5$ or $y^5$ term all over 5. 
		THe RHS is of the same format: binomial expansion of $(-x-y)^7$ without $x^7, y^7$ term divided by 7.

		Now by symmetry, we can easily see that the two equations are equal (if you don't trust me, just expand it out).
	\end{enumerate}
\end{Exercise}

\begin{Exercise}
	This is a hard figure to draw, so I'll introduce some notation for me to not draw it. 
	We'll use ordered pair notation, where the first coordinate is which row from the bottom, and the second coordinate is which column from the left.
	For example, $(0,0)$ would be the bottom-right most penny, $(0,1)$ would be to the right of it.
	Now $(1,0)$ would be the first penny in the second row meaning that $(5,0)$ is the only element with 5 in the first coordinate (since there's only one penny on the fifth row).
	Be aware this is zero-indexing.

	We will prove by contradiction and let's examine the inner-most triangle $(1,1), (1,2), (2,1)$. 
	WLOG let's say that $(2,1)$ is of the color white, and the other two points are black by pigeon-hole principal.
	This then forces $(0,2)$ to be also white as if it's black, then $(0,2), (1,1), (1,2)$ will form a triangle.

	Now note that $(1, 0), (1, 3)$ MUST be black also as there's a equilateral triangle with the color white with $(1,0),(2,1),(0,2)$ (and by symmetry, the other one too).
	Finally, if we look at $(0,1)$ or $(0,3)$, they can't be black or white! 
	Thus, contradiction and we are done. \qed
\end{Exercise}