\setcounter{Exercise}{3}
\begin{Exercise}
	\begin{enumerate}[a]
		\item We split into cases. For both $x, y$ positive, it is obvious that this is true with equality.
		Same holds for both numbers being negative. 
		Also note that if one or both numbers are 0, then it is again trivial. 

		The only non-trivial aspect of the problem is the $x > 0 > y$ case WLOG.
		In this case, then we have
		\begin{align*}
			x - y > (x + y) \\
			x - y > -x + (-y) > -(x+y)
		\end{align*}
		which are the two cases for $|x + y|$.

		\item If we have part (a), then we can substitute $x-z$ and $y-z$ for $x, y$ in (a).
	\end{enumerate}
\end{Exercise}

\begin{Exercise}
	Began by noting that if the denominators are both positive, then the inequality will never be true. 
	For simplicity, let's break up the problem into two cases:
	\begin{enumerate}
		\item $x-1 < 0 < x+1 \rightarrow -1 < x < 1$
		\item $x-1 < x + 1 < 0$.
	\end{enumerate}

	Let's consider the second case first, which is easily solved
	\begin{align*}
		\frac{3}{x-1} &< \frac{2}{x+1} \\
		3x + 3 &< 2x - 2 \\
		x &< -5
	\end{align*}

	As for the first case, just be careful to flip the sign
	\begin{align*}
		\frac{3(x+1)}{x-1} &< 2 \\
		3x + 3 &> 2x - 2 \\
		x &> -5
	\end{align*}
	and note that we have the constraint that $|x| < 1$. \qed
\end{Exercise}

\begin{Exercise}
\end{Exercise}
\begin{Exercise}
\end{Exercise}

\begin{Exercise}
	Let's start with working out the base cases of 0 and 1.
	They are $F(1)^2 - F(1) = 2$ and $F(0)^2 - F(0) = 0$.
	Obviously, there's two answers for each of the equations above, but checking each case reveals that $F(1) = 2, F(0) = 1$.

	Now it's easy to prove that $F(x)=x+1$ for all integers by $F(x)F(1) - F(x) = 2F(x) - F(x) = F(x) = x + 1$.
	Heck, using this argument it extends to all the reals! \qed
\end{Exercise}